\section{Age of Sigmar Cheat Sheet\\ \Large{Daughters of Khaine}}

\begin{itemize}
\item Setup: 
    \begin{itemize}
        \item If your general is a hero, pick
            a \hyperref[command-traits]{Command Trait}\marginpar{Command Traits
            are passive buffs}
        \item The \textsc{hero} can pick an
            \hyperref[artefacts-of-power]{Artefact of Power} from the
            \hyperref[gifts-of-morathi]{\textsc{Gifts of Morathi}}.
            A \textsc{Wizard} can take an
            \hyperref[artifact-of-shadows]{\textsc{Artifact of Shadow}} and
            a \textsc{priest} can take from the
            \hyperref[relics-of-khaine]{\textsc{Relics of Khaine}}. Pick one
            \emph{additional} artifact for each warscroll battalion.
        \item \emph{Each} \textsc{Wizard} knows one \emph{additional} spell
            from the \hyperref[lore-of-shadows]{Lore of Shadows} and each
            \textsc{priest} one \emph{additional} prayer from the
            \hyperref[prayers-of-the-khainite-cult]{Prayers of the Khainite
            Cult.}
        \item Pick a \hyperref[temple]{temple} to get the temple benefits (and
            the restrictions)
    \end{itemize}

\item Basic troops (Witch Aelves and Sisters of Slaughter) have the ability to
    run and charge with only a musician.  You can reroll 1's to run turn 1, 1's
        to charge turn 2\marginpar{Even improved by
        \hyperref[blood-rituals]{Blood Rituals \textbar{} Warcoven}}, plus Deep
        Striking Khinerai, Khailebron's deep striking ability and Lore of
        Shadows spells that boost movement.
\item Saving Throws:
    \begin{itemize}
        \item \hyperref[bloodshield]{Bloodshield} of Cauldron: Add 1 to saving
            throws for friendly \textsc{DoK} units that are wholly within this
            \marginpar{Optimized you can get a +2 on save rolls and re-roll
            save-after-save with 5+ drastically improving the saves}
            range of this model.
        \item \hyperref[witchbrew]{Witchbrew} improves saving throws by 1. Witchbrewed
            units do not take a battleshock test.
        \item \hyperref[fanatical-faith]{Fanatical Faith:} Daughters of Khaine
            can ignore wounds allocated to them on a 6+. Reroll with
            \hyperref[blessing-of-khaine]{Blessings of Khaine}.
        \item \hyperref[blessing-of-khaine]{Blessing of Khaine} allow re-roll
            of failed \hyperref[fanatical-faith]{Fanatical Faith} rolls for that unit
        \item \hyperref[hagg-nar]{Hagg Nar} command trait (passive)
            \hyperref[devoted-disciples]{Devoted Disciples}: Wound is negated
            on 5+ instead of 6+ within 7 inches of general. 
        \item Morathi: \hyperref[enchanting-beauty]{Enchanting Beauty}
        \item Shooting at \textsc{Characters} (\hyperref[corerule:look-out-sir]{Look
            out!}) give the enemy -1 to hit unless the target is a monster.
            Cauldron of Blood \marginpar{Keep them close to a unit and watch
            them get to combat unmolested} or the Bloodwrack Shrine do not
            count as Monsters. 
        \item \hyperref[the-iron-heart-of-khaine]{The Iron Heart of Khaine:
            Morathi}, High Oracle of Khaine cannot be healed, but no more than
            3 wounds can be allocated to her in any one turn. Any additional
            wounds and/or mortal wounds allocated to her in the same turn are
            negated and have no effect.
        \item When \hyperref[war-coven]{War Coven} has been chosen as warscroll: No battleshock tests
            within 18~inch of \textsc{Morathi} due to \hyperref[devout-followers]{Devout
            Followers}
        \item Your enemy gets -1 to hit in the shooting phase against
            Khailebron units
        \item \hyperref[unquenchable-fervour]{Unquenchable Fervour} in phase 5
    \end{itemize}
\end{itemize}


\newpage %---------------------------------------------------------------
\section{Gamephases}
\begin{enumerate}
    \item{\textbf{\htarget{hero-phase}{Hero Phase}}}\marginpar{Only abilites
        that can be used in the phase are listed.}

    \begin{description}[align=left]
        \item [Command] 
            Heros may spend a command point. Remember additional (passive)
            \hyperref[command-traits]{command traits} abilities if general is
            a hero.
            \begin{itemize}
            \item If Morathi, High Oracle of Khaine is your general\marginpar{Do not
            make her general if you want to teleport her into battle.
            \hyperref[temple:khailebron]{Khailebron}} you can use the command
            ability \hyperref[worship-through-bloodshed]{Worship Through
            Bloodshed}
            \item Khailebron temple \hyperref[mistress-of-illusion]{deepstrike} ability
            \item Battalion abilities such as \hyperref[righteous-fervour]{Righteous Fervour}
            \end{itemize}


        \item [\textsc{wizard} and \textsc{priest} spells] Remember one
            additional spell from \hyperref[lore-of-shadows]{Lore of Shadows}
            and one additional prayer from
            \hyperref[prayers-of-the-khainite-cult]{Prayers of the Khainite
            Cult} These units count as wizards:\marginpar{Hag Queen is a priest
            but can unbind spells.}
            \begin{itemize}
                \item \hyperref[morathi-wizard]{Morathi}
                    \hyperref[sorceress-supreme]{Sorceress Supreme}:
                    Add 1 to casting and unbinding rolls for Morathi, High
                    Oracle of Khaine. Double the range of spells she attempts
                    to cast.
                \item \hyperref[morathi-the-shadow-queen]{Morathi, Shadow Queen}
                    \begin{scriptsize}
                        \hyperref[spell:arnzipals-black-horror]{Arnzipal’s Black Horror},
                        One from the \hyperref[lore-of-shadows]{Lore of Shadows}
                    \end{scriptsize}
                \item \hyperref[bloodwrack-shrine]{Bloodwrack Shrine} and
                        \hyperref[bloodwrack-medusa]{Bloodwrack Medusa}
                    \begin{scriptsize}
                        \hyperref[spell:enfeebling-foe]{Enfeebling Foe}
                    \end{scriptsize}
                \item \hyperref[doomfire-warlocks]{Doomfire Warlocks}
                    \begin{scriptsize}
                        \hyperref[doomfire]{Doomfire}
                    \end{scriptsize}
            \end{itemize}
            These unit count as priests:
            \begin{itemize}
                \item \hyperref[hag-queen]{Hag Queen}\marginpar{Priestess of
                        Khaine: Roll 1D. Mortal wound on one, nothing happens on
                        two, 3+ prayer successful}
                    \begin{scriptsize}
                        \hyperref[rune-of-khaine]{Rune of Khaine},
                        \hyperref[touch-of-death]{Touch of Death}
                    \end{scriptsize}
                \item \hyperref[slaughter-queen]{Slaughter Queen}
                    \begin{scriptsize}
                        \hyperref[rune-of-khaine]{Rune of Khaine},
                        \hyperref[touch-of-death]{Touch of Death},
                        \hyperref[dance-of-doom]{Dance of Doom}
                        \hyperref[orgy-of-slaughter]{Orgy if Slaughter} if she
                        is the \textsc{General}
                    \end{scriptsize}
                \item \hyperref[hag-queen-on-cauldron]{Hag Queen on Cauldron}
                    \begin{scriptsize}
                        \hyperref[rune-of-khaine]{Rune of Khaine}, 
                        \hyperref[touch-of-death]{Touch of Death}, 
                        \hyperref[wrath-of-khaine]{Wrath of Khaine}, 
                        \hyperref[idol-of-worship]{Idol of Worship}
                    \end{scriptsize}
                \item \hyperref[slaughter-queen-on-cauldron-of-blood]{Slaughter
                    Queen on Cauldron}
                    \begin{scriptsize}
                        \hyperref[rune-of-khaine]{Rune of Khaine}, 
                        \hyperref[touch-of-death]{Touch of Death}, 
                        \hyperref[wrath-of-khaine]{Wrath of Khaine} 
                        \hyperref[orgy-of-slaughter]{Orgy if Slaughter} if she
                        is the \textsc{General}
                    \end{scriptsize}
            \end{itemize}

        \item [Abilities] played in the hero phase:
            \begin{itemize}
                \item Hag Queen 
                    \begin{scriptsize}
                        \hyperref[witchbrew]{Witchbrew} 
                        \marginpar{Never forget to brew!}
                    \end{scriptsize}
                \item Doomfire Warlocks
                    \begin{scriptsize}
                        \hyperref[doomfire-coven]{Doomfire Coven} 
                    \end{scriptsize}
                \item Medusa (also on Cauldron)
                    \begin{scriptsize}
                        \hyperref[aura-of-agony]{Aura of Agony} 
                    \end{scriptsize}
            \end{itemize} 
            
    \end{description}

    \item{\textbf{Movement Phase}}\hypertarget{movementphase}{}
    %---------------------------------------------------------------
    \begin{description}
      \item [Shadow Patrol] Deepstrike ability \hyperref[shadowpaths]{Shadowpaths}
      \item [Shadowhammer] \hyperref[righteous-fervour]{Righteous Fervour}
      \item [Khinearai Heartrenders and Lifetakers]
          \hyperref[descend-to-battle]{Descend to Battle}
      \item [Blood Rites] \hyperref[quickening-bloodlust]{Quickening Bloodlust}:
          Reroll run rolls of 1
    \end{description}


    \item{\textbf{Shooting Phase}}\hypertarget{shootingphase}{}
    %---------------------------------------------------------------
    \begin{description}
        \item \textsc{\textbf{Khinerai Heartrenders}} Fire and Flight: In your shooting phase,
          after this unit has finished making all of its attacks, roll a dice:
            on a 4+ it can make a 6~inch normal move as if it were your
            movement phase, but it cannot retreat or run as part of this move.
    \end{description}

    \begin{itemize}
    \tightlist
    \item
      \textbf{Gaze of Morathi}: If a target is hit by the Gaze of Morathi,
      pick a model in the target unit and roll a dice. If the result exceeds
      that model'sa Wounds characteristic, it is slain. 
    \end{itemize}

    %---------------------------------------------------------------
    \item{\textbf{Charge Phase}}
        \begin{itemize} 
            \item Cauldron \textbar{} Bladed Impact: Roll a dice if this model
                ends a charge move within 1~inch of any enemy units. On a 2+
                the nearest enemy unit suffers D3 mortal wounds.
            \item Blood Rites (Phase 2): \hyperref[headling-fury]{Headlong Fury}
            \item Blood Rites (Phase 3): \hyperref[zealots-rage]{Zealot's
                Rage}\marginpar{Hagg Nar units: Re-Roll all failed hit rolls
                due to \hyperref[first-temple]{Daughters of the First Temple}}
        \end{itemize}

    %---------------------------------------------------------------
    \item{\textbf{Combat Phase}}
        \begin{itemize}
            \item \hyperref[witchbrew]{Witchbrew}
            \item Blood Rites (Phase 3): \hyperref[zealots-rage]{Zealot's Rage}
            \item Blood Rites (Phase 4):
                \hyperref[slaughterers-strength]{Slaughterer's Strength}
        \end{itemize}
    \item{\textbf{Battleshock Phase}} Witchbrewed? Ignore Battleshock!
\end{enumerate}


\hypertarget{alligence-abilities}{%
    \section{Alligence Abilities}\label{alligence-abilities}}

\hypertarget{battle-traits}{%
    \subsubsection{Battle Traits}\label{battle-traits}}
\begin{description}[align=left]
    \item [\htarget{fanatical-faith}{Fanatical Faith}] Ignore wounds allocated to them on a 6+
\item [Blood Rites] Cummulative effects for each round
  \begin{enumerate}
          \marginpar{War Coven of Morathi treat current battle round one higher
              with \hyperref[blood-rituals]{Blood Rituals}}
  \item \textbf{\htarget{quickening-bloodlust}{Quickening Bloodlust}} Reroll
      run rolls of 1.
  \item \textbf{\htarget{headling-fury}{Headlong Fury}} Reroll dice rolls of
      1 when charging.
  \item \textbf{\htarget{zealots-rage}{Zealot's Rage}} Rerolls 1's to hit. In
      addition, an Avatar of Khaine always counts as being animated.
          \marginpar{Hagg Nar units can re-roll \emph{all} failed hit rolls}
  \item \textbf{\htarget{slaughterers-strength}{Slaughterer's Strength}} Reroll
          \marginpar{\hyperref[witchbrew]{Witchbrew} lets you re-roll all failed to wound rolls}
      1's to wound.
  \item \textbf{\htarget{unquenchable-fervour}{Unquenchable Fervour}} Your
      units rerolls saves of 1, and do not need to take battleshock tests.
  \end{enumerate}
\end{description}

\hypertarget{command-traits}{%
    \subsection{Command Traits}\label{command-traits}}
If the general of a \textsc{DoK} army is a hero, they can have
one of the following command traits. in addition to any others they
have: 
\begin{description} 
\item [Bathed in Blood] Increase your generals's wound characteristic by~1 and you 
  can heal 1~wound at the start of each hero phase. Good on a Cauldron of Blood, but 
  there are better ways to get your wounds back.
\item [Zealous Orator] Friendly \textsc{DoK} units
  within 14~inch of this general use this general's Bravery characteristic
  instead of their own.
\item [Bloody Sacrificer] Add~1 to hit rolls for this
  general's weapons.
\item [Terrifying Beauty] Subtract~1 from the hit rolls of
  attacks that target this general.
\item [{\htarget{mistress-of-poisons}{Mistress of Poisons}}] Add~1 to the Damage
  characteristics of melee weapons wielded by this general.
\item [True Believer] This general counts the current battle
  \marginpar{Improves the turn number for the unit's Blood Rites. Good to get
        an Avatar of Khaine awake turn~2 and just generally a good choice.}
  round number as being 1~higher than it actually is, when determining
  what abilities they receive from the Blood Rites battle trait.
  This is cumulative with other, similar abilities.
\end{description}

%---------------------------------------------------------------
\hypertarget{lore-of-shadows}{%
    \subsection{Lore of Shadows}\label{lore-of-shadows}}
Each \textsc{wizard} in a \textsc{DoK} army knows one spell from the Lore
of Shadows in addition to any others they know.\\
\marginpar{Note that Doomfire Warlocks are Wizards and can take a spell from here despite not 
being a Hero unit.}
\begin{description}
\item [\htarget{steed-of-shadows}{Steed of Shadows}] has a casting value of
  5. If successfully cast, then until the start of your next hero phase,
  the caster can fly and has a Move characteristic of 16~inch.
\item [\htarget{pit-of-shades}{Pit of Shades}] has a casting value of
    \marginpar{Inflict 2W6$-$\textsc{Move} number of wounds}
  7. If successfully cast, pick an enemy unit within 18~inch of the caster
  that is visible to them. Roll two dice and add the scores together.
  The enemy unit suffers 1 mortal wound for each point by which the
  total exceeds their Move characteristic.
\item [\htarget{mirror-dance}{Mirror Dance}] has a casting value of 4.
    \marginpar{If you have any Hagg Nar \textsc{Wizards} in your army, Mirror
        Dance is great, as it is able to transport your general to wherever
        they are needed the most to protect and boost your troops.}
  If successfully cast, pick two friendly \textsc{DoK} heros
  within 24~inch of the caster. So long as neither \textsc{hero} is within 6~inch of any
  other unit, the two models can swap positions on the battlefield
  (neither can be set up within 3~inch of any enemy units).
\item [\htarget{the-withering}{The Withering}] has a casting value of
  7. If successfully cast, pick an enemy unit within 18~inch of the caster
  that is visible to them. Until the start of your next hero phase, add
  1 to wound rolls for attacks that target that unit.
\item [\htarget{mindrazor}{Mindrazor}] has a casting value of 7. If
    \marginpar{If lower bravery characteristics, then inflict +1 wound}
  successfully cast, pick a friendly \textsc{DoK} unit within 18~inch
  of the caster. Until the start of your next hero phase, the Rend
  characteristic of that unit's melee weapons is improved by 1. In
  addition, the Damage characteristic of the unit's melee weapons is
  increased by 1 while attacking a target that has a lower Bravery
  characteristic than they do.
\item [\htarget{shroud-of-despair}{Shroud of Despair}] has a casting
  value of 4. If successfully cast, pick an enemy unit within 18~inch of the
  caster that is visible to them. Until the start of your next hero
  phase, subtract 1 from the Bravery characteristic of that unit. If the
  spell was successfully cast with a casting roll of 8 or more, subtract
  D3 from that unit's Bravery instead.
\end{description}

\hypertarget{prayers-of-the-khainite-cult}{%
    \subsection{Prayers of the Khainite Cult}\label{prayers-of-the-khainite-cult}}
\marginpar{A model that knows such a prayer can pray twice in your hero phase
instead of only once (but not the same spell)}
Each \textsc{priest} in a \textsc{DoK} army knows one prayer from the six
Prayers of the Khainite Cult in addition to any others they know. 
\begin{description}
    \item [\htarget{catechism-of-murder}{Catechism of Murder}] Pick a friendly
        \textsc{DoK} unit within 14~inch of the priest. Until the start of your next
        hero phase, each time you make a hit roll of 6 (after re-rolls, but
        before modifiers are applied) for that unit in the combat phase, that
        \marginpar{Exploding sixes}
        attack inflicts 2 hits instead of 1.
    \item [\htarget{blessing-of-khaine}{Blessing of Khaine}] Pick a friendly
        \marginpar{Reroll all missed 6+ saving throws!}
        \textsc{DoK} unit within 14~inch of the priest. Until the start of your next
        hero phase, re-roll failed \hyperref[fanatical-faith]{Fanatical Faith}
        rolls for that unit.
    \item [\htarget{martyrs-sacrifice}{Martyr's Sacrifice}] Pick a friendly \textsc{DoK}
        \marginpar{Think twice before attacking your Witch Elves}
        unit within 14~inch of the priest. Until the start of your next hero
        phase, each time a model from that unit is slain in the combat phase,
        roll a dice. On a 5 or 6 the attacking unit suffers 1 mortal wound
        after it has finished making all of its attacks.
    \item [\htarget{crimson-rejuvenation}{Crimson Rejuvenation}] Pick
        \marginpar{One of few healing possibilites DoK have.}
        a friendly \textsc{DoK} unit within 14~inch of the priest (Except Morathi). You
        can heal up to D3 wounds that have been allocated to a model from that
        unit.
    \item [\htarget{covenant-of-the-iron-heart}{Covenant of the Iron Heart}]
        Pick a friendly \textsc{DoK} unit within 14~inch of the priest. Until the start
        of your next hero phase, you do not need to take battleshock tests for
        that unit.
    \item [\htarget{sacrament-of-blood}{Sacrament of Blood}] Pick a friendly
        \textsc{DoK} unit within 14~inch of the priest. Until the start of your next
        hero phase, that unit counts the current battle round number as being
        1 higher than it actually is when determining what abilities it
        receives from the Blood Rites battle trait. This is
        cumulative with other, similar abilities.
\end{description}

\hypertarget{artefacts-of-power}{%
    \subsection{Artefacts of Power}\label{artefacts-of-power}}
If a \textsc{DoK} army includes any heros, then one may bear an
artefact of power.\\
If your army includes any \hyperref[draichi-ganeth]{Draichi Ganeth} Slaughter
Queens, one must have the \textsc{darksword}.
One Kraith \textsc{hero} must have the artefact of power
\hyperref[venom-of-nagendra]{Venom of Nagendra}

\subsubsection{{\htarget{gifts-of-morathi}{Gifts of Morathi}}}
\begin{description} 
    \item [\htarget{crown-of-woe}{Crown of Woe}] Subtract 1 from their Bravery
        characteristic of enemy units that are within 7~inch of the bearer. The
        first time the bearer slays an enemy model, the range of this ability
        is increased to 14~inch for the remainder of the battle.
    \item [\htarget{cursed-blade}{Cursed Blade}] Pick one of the bearer’s melee
        weapons. Add 1 to hit rolls made for that weapon. In addition, each
        time a hit roll of 7+ is made for that weapon, the target suffers
        1 mortal wound instead of the normal damage.
    \item [\htarget{amulet-of-dark-fire}{Amulet of Dark Fire}] Roll a dice each
        time the bearer is allocated a mortal wound that was inflicted by an
        enemy spell. On a 4+ that wound is negated.
    \item [\htarget{crone-blade}{Crone Blade}] Pick one of the bearer’s melee
        weapons. Each time an enemy model is slain by an attack made with this
        weapon, you can heal 1 wound that has been allocated to the bearer.
    \item [\htarget{thousand-and-one-dark-blessings}{Thousand and One Dark
        Blessings}] Add 1 to save rolls for the bearer.
    \item [\htarget{bloodbane-venom}{Bloodbane Venom}] Pick one of the bearer’s
        melee weapons. If a model is allocated any wounds from attacks made
        using that weapon but is not slain, roll a dice after the bearer has
        finished making all of their attacks. If the roll equals or exceeds
        that model’s Wounds characteristic, it is slain.
\end{description} 

\subsubsection{\htarget{artifact-of-shadows}{Artefacts of Shadow}} One
\textsc{hero} can take an artifact of power. \textsc{Wizards} can take an
\textsc{Artifact of Shadows} and \textsc{Priests} can take a \textsc{Relict of
Khaine}.
\begin{description} 
    \item [\htarget{shadow-stone}{Shadow Stone}] Re-roll dice rolls of 1 that
        are made as part of a casting roll for the bearer. In addition, add
        1 to the casting roll if the bearer attempts to cast a spell from the
        Lore of Shadows
    \item [\htarget{rune-of-ulgu}{Rune of Ulgu}] The bearer knows one
        additional spell from the Lore of Shadows 
    \item [\htarget{the-mirror-glaive}{The Mirror Glaive}] Each time the bearer
        unbinds an enemy spell, they can immediately attempt to cast either the
        \hyperref[mystic-shield]{Mystic Shield} or
        \hyperref[arcane-bolt]{Arcane Bolt} spells as if it were your hero
        phase. Your opponent cannot attempt to unbind this spell if the casting
        roll is
        successful.
    \item [\htarget{seven-fold-shadow}{Seven-fold Shadow}] Once per battle,
        instead of moving the bearer in your movement phase, you can remove
        them from the battlefield and set them up anywhere on the battlefield
        more than 9~inch from any enemy models. This is their move for that
        movement phase.
    \item [\htarget{crystal-heart}{Crystal Heart}] The bearer can attempt to
        cast a second spell in each of your hero phases. If they do so, roll
        a dice before the casting roll is made. On a 1, the bearer suffers D3
        mortal wounds.
    \item [\htarget{shade-claw}{Shade Claw}] The bearer’s Whisperclaw has
        a Rend characteristic of -2.
\end{description} 

\subsubsection{\htarget{relics-of-khaine}{Relics of Khaine}}
\begin{description} 
    \item [\htarget{blood-sigil}{Blood Sigil}] The bearer knows one additional
        prayer from the Prayers of the Khainite Cult 
    \item [\htarget{iron-circlet}{Iron Circlet}] Whenever the bearer prays,
        re-roll rolls of 1 when seeing if the prayer is successful or not.
    \item [\htarget{rune-of-khaine}{Rune of Khaine}] When the bearer is slain,
        roll a dice. On a 1 nothing happens. On 2--5 the unit that slew them
        suffers D3 mortal wounds. On a 6 the unit that slew them suffers D6
        mortal wounds.
    \item [\htarget{crimson-shard}{Crimson Shard}] The bearer’s Blade of Khaine
        has a To Wound characteristic of 2+.
    \item [\htarget{khainite-pendant}{Khainite Pendant}] The bearer can pray
        three times in your hero phase. However, the first time a 1 is rolled
        when the bearer prays and they are found unworthy, they suffer D3
        mortal wounds instead of 1.
    \item [\htarget{hagbrew}{Hagbrew}] Add 1 to wound rolls for the bearer’s
        melee weapons.
\end{description} 


\hypertarget{temple}{%
    \subsection{Temple}\label{temple}}
\subsubsection{\htarget{hagg-nar}{\textsc{Hagg Nar}}}
Your general's command trait \emph{must} be Devoted Diciples (passive ability):
\begin{description} 
    \item [\htarget{devoted-disciples}{Devoted Disciples}] (Command Trait)
        Whenever you make a \hyperref[fanatical-faith]{Fanatical Faith} roll
        for a friendly Hagg Nar unit within 7~inch of this general, the
        wound is negated on a 5+ instead of a 6+.  
    \item [\htarget{first-temple}{Daughters of the First Temple}] Whilst a Hagg
        Nar unit is benefitting from the \hyperref[zealots-rage]{Zealot's Rage}
        ability from the Blood Rites battle trait, you can re-roll all failed
        hit rolls for the unit instead of only re-rolling hit rolls of 1.
    \item [Warscroll Building] A Hagg Nar \hyperref[cauldron-guard]{Cauldron
        Guard} battalion can also include 1~Avatar of Khaine or an additional
        \textsc{Cauldron of Blood}
\end{description}
A \textsc{Cauldron of Blood} makes for an exceptional Hagg Nar general, acting
as a lynchpin and force-multiplier for the entire army. The combination of its
\hyperref[bloodshield]{Bloodshield} ability and the
\hyperref[devoted-disciples]{Devoted Disciples} command trait grants your
fragile Daughters of Khaine much-needed protection. Consider the
\hyperref[thousand-and-one-dark-blessings]{Thousand and One Dark Blessings}
artefact for this general too, making a tough model even more resilient. If you
have other battalions, the \hyperref[iron-circlet]{Iron Circlet} or
\hyperref[khainite-pendant]{Khainite Pendant} will ensure your prayers are
answered, and the Avatar of Khaine atop the cauldron can awaken and add its
attacks to the fray. Otherwise, make sure you also have another \textsc{Priest}
or two, such as a Hag Queen. Remember that the \hyperref[wrath-of-khaine]{Wrath
of Khaine} prayer is not one of the six Prayers of the Khainite cult, so can be
attempted by more than one \textsc{Priest} in the same turn (besides, if you
have several Avatars of Khaine, you will need several \textsc{Priest} to awaken
them). That way you can also use the \hyperref[crimson-rejuvenation]{Crimson
Rejuvenation} prayer to mitigate any wounds your general does suffer, and
\hyperref[sacrament-of-blood]{Sacrament of Blood} to ensure your units get the
benefit of their First Daughter of Khaine ability that much quicker. Finally,
if you have any Hagg Nar \textsc{Wizard} in your army,
\hyperref[mirror-dance]{Mirror Dance} is great, as it is able to transport your
general to wherever they are needed the most to protect and boost your troops.
\\

By turn 3 your army is a killing machine and would be already in combat. This is
when things swing dramatically for Hagg Nar. Any unit with just a simple
\hyperref[witchbrew]{witch brew} and being Hagg Nar will be rerolling their
hits and wounds. On a unit of 20 witches that is 80 attacks (reroll hits and
wounds) and if you add \hyperref[mindrazor]{Mindrazor} into the mix no single opponent
can survive that number of paper cuts. 


\subsubsection{\htarget{draichi-ganeth}{\textsc{Draichi Ganeth}}}
If your army includes any Draichi Ganeth \hyperref[slaughter-queen]{Slaughter
Queen's} one must have the artifact \textbf{\htarget{darksword}{The
Darksword}}. This Slaughter Queen's Deathsword has an attacks characteristic
of~4. A Slaughter Troupe may take up to two units of Witch Elves (hitting on
2's, rerolling 1's on turn~3 will make almost every attack hit).\\
\begin{description}
    \item [\htarget{bladed-killers}{Bladed Killers}] Add~1 to hit rolls for Draichi
    Ganeth units in the combat phase if they charged in the same turn.
\end{description}
Draichi Ganeth \hyperref[blood-sisters]{Blood Sisters} are very good
\marginpar{Snake heavy armies take note --- combine with Hag buffs for Crystal
    Vision mortal wounds on 3+, 6's count as two and reroll 1's. That's some scary
    MW output before the sneks start swinging.}
shock-assault units, able to scythe through lesser infantry with ease and slay
tough foes using their crystal touch. They are lethal on the charge, their
Bladed Killers ability allowing you to add 1 to hit rolls for them, meaning
that they can inflict mortal wounds on dice rolls of 3+. They should always be
accompanied by a Draichi Ganeth Slaughter Queen, preferably one on foot so that
she can keep pace with them.  The Catechism of Murder prayer means that any hit
rolls of 6 made for the Blood Sisters inflict 2 hits -- that’s potentially
2 mortal wounds from each model.  The Slaughter Queen, with her Deathsword, can
then use her Dance of Doom prayer to make up to 8 attacks dealing D3 damage
each. In the next turn, use her Orgy of Slaughter ability on the Blood Sisters
to allow them to fight again, hopefully freeing them up so that they can move
and charge a second unit. They will now perform even better than they did in
the previous battle round because of the extra benefit they will have gained
from the Blood Rites battle trait.

\subsubsection{\textsc{The Kraith}} 
Several units of Kraith Witch Aelves can decimate the enemy, especially when
used alongside a Kraith \textsc{Wizard} who casts Mindrazor. The high number of
attacks that the Witch Aelves can make maximises the damage potential of this
combination, so long as you target units with a lower Bravery characteristic.
Using such a unit alongside a Kraith general with the Zealous Orator command
trait and/or an Avatar of Khaine will mean they will certainly be braver than
any Battleline unit they face. If combined with a second \textsc{Hero} bearing
the Crown of Woe, or a \textsc{Wizard} casting Shroud of Despair, then even
elite units need to beware. The reason this tactic works so well with the
Kraith is their Disciples of Slaughter ability; if after attacking you roll
a 6, then you get to fight with your boosted unit all over again. However, this
unit of Witch Aelves will become a priority target for your opponent – they
will do everything in their power to destroy it before it rampages through
their lines. If a Kraith \textsc{Priest} can use the Martyr’s Sacrifice on the
unit then your opponent has a dilemma: even if they do attack first, there’s
still a good chance that the Witch Aelves will inflict significant damage in
return.

One Kraith hero must have the artifact Venom of Nagendra:\\
\marginpar{Not as useful as Hagg Nar or Draichi Ganeth (which help you win the
fights you already got into) or Khailebron (which helps you get to the fight in
one piece).} 

\textbf{\htarget{venom-of-nagendra}{Venom of Nagendra}}.  
Once per battle, just before this hero is chosen to fight in the combat phase,
she can use the Venom of Nagendra. When she does so, choose one of her melee
weapons (but not a weapon used by a mount). That weapon's Attacks
characteristic is 1 for the remainder of the phase, but if it hits, the target
suffers D6 mortal wounds instead of the normal damage.\\ 
Cauldron Guard may take any number of Hag Queens and Slaughter Queens. 
\begin{description}
    \item [\htarget{disciples-of-slaughter}{Disciples of Slaughter}] Roll
        a dice after a Kraith unit has fought in the combat phase if there are
        any enemy units within 3~inch of it. On a 6, you can pile in and attack
        with that unit for a second time.
\end{description}

\subsubsection{\textsc{Khailebron}}\label{temple:khailebron}
Whilst deadly at range, Blood Stalkers are vulnerable to any return fire, but
the Khailebron’s Concealment and Stealth ability will help to keep them alive
whilst they slay the enemy from afar. To get even more out of their shooting,
team them up with a PRIEST to use Sacrament of Blood in order to access the
Zealot’s Rage or Slaughterer’s Strength abilities from the Blood Rites table as
soon as possible. Alternatively, use a Khailebron WIZARD to cast the Withering
on your victims and watch that bow fire reap a bloody tally.  If Khailebron
Blood Stalkers are taken as part of a Temple Nest battalion, then to eliminate
the threat they pose your opponent will need to close the distance to negate
their Concealment and Stealth, at which point they are forced into your
Bloodwrack Shrine’s Aura of Agony – which they would otherwise be trying hard
to avoid. If you happen to have Morathi in your army, Blood Stalkers are good
targets for her Worship Through Bloodshed ability, especially on turns 1 and 2,
when they are unlikely to be in combat and Morathi has yet to transform into
the Shadow Queen.

Your enemy gets -1 to hit in the shooting phase against Khailebron units and
your General's command must be Mistress of Illusion.\\
Temple Nest may take to two additional units of Blood Sisters or Blood Stalkers
(or mix). This plugs one of your army's major weaknesses (getting shot to
pieces)\marginpar{Take this against shooting armies} and the Command Trait can
lead to some fascinating tactics, where your opponent never feels safe.
\begin{description}
  \item [Concealment and Stealth] Subtract 1 from hit rolls 
    that target Khailebron units in the shooting phase.
  \item [Warscroll Building] 
    A Khailebron Temple Nest battalion can include up to
    2 additional MELUSAI (Blood Sisters or Blood Stalkers) units.
\item [\htarget{mistress-of-illusion}{Mistress of Illusion}]
    \marginpar{Deepstrike} At the start of your hero phase, you can pick
        a friendly Khailebron unit within 7~inch of this general. If that unit
        is more than 3~inch from any enemy models, remove it from the
        battlefield and then set it up anywhere on the battlefield more than
        9~inch from any enemy models. The unit cannot move in your next movement
        phase. 
\end{description}

\newpage
\hypertarget{Warscroll}{%
    \section{Warscroll}\label{warscroll}}
\subsection{\textsc{War Coven of Morathi}}
The \hypertarget{war-coven}{War Coven} of Morathi consists of the following
units and warscroll battalions:
\begin{itemize}
    \tightlist
    \item Morathi, High Oracle of Khaine
    \item 1 \hyperref[cauldron-guard]{Cauldron Guard}
    \item 1 \hyperref[slaughter-troupe]{Slaughter Troupe}
    \item 1 \hyperref[temple-nest]{Temple Nest}
    \item 1 \hyperref[shadow-patrol]{Shadow Patrol}
\end{itemize}
\begin{description}
    \item [\htarget{blood-rituals}{Blood Rituals}] If your army has the
        \textsc{DoK} allegiance, units in this battalion count the
        current battle round number as being 1 higher than it actually is when
        determining what abilities they receive from the Blood Rites battle
        trait. This is cumulative with other, similar abilities
        (e.g.~the True Believer Command Trait or the Sacrament of Blood
        prayer).  \item [\htarget{devout-followers}{Devout Followers}] Do not
            take a battleshock test for War Coven of Morathi units that are
            within 18~inch of Morathi (in either of her forms) when the test is
            taken. 
\end{description}
        
\subsection{\textsc{Cauldron Guard}}
Requires the following organisation of units:
\begin{itemize}
    \tightlist
    \item 1 Hag Queen
    \item 2 units of Witch Aelves
    \item 2 units of Khinerai Lifetakers
\end{itemize}
\begin{description}
    \item [\htarget{frenzied-devotees}{Frenzied Devotees}] Add 1 to run and
        charge rolls made for units from this battalion. 
\end{description}
        
\subsection{\textsc{Slaughter Troupe}}
Requires the following organisation of units:
\begin{itemize}
    \tightlist
    \item 1 Slaughter Queen
    \item 2 units of Sisters of Slaughter
    \item 2 units of Khinerai Heartrenders
\end{itemize}
\begin{description}
    \tightlist
    \item [\htarget{gladiatorial-acrobatics}{Gladiatorial Acrobatics}]
        Slaughter Troupe units that retreat can still shoot and charge in the
        same turn. 
\end{description}

\subsection{\textsc{Temple Nest}}
Requires the following organisation of units:
\begin{itemize}
    \tightlist
    \item 1 Bloodwrack Medusa
    \item 2 units of Blood Stalkers
    \item 2 units of Blood Sisters
\end{itemize}
\begin{description}
    \tightlist
    \item [\htarget{lethal-transfixion}{Lethal Transfixion}] Each time your
        opponent makes a hit roll of 1 when attacking a Temple Nest unit in the
        combat phase, the attacking unit suffers 1 mortal wound after all of
        its attacks have been made. 
\end{description}

\subsection{\textsc{Shadow Patrol}}
Requires the following organisation of units:
\begin{itemize}
    \tightlist
    \item 2 units of Doomfire Warlocks
    \item 4 units of Khinarai Harpies
\end{itemize}
\begin{description}
    \tightlist
    \item [\htarget{shadowpaths}{Shadowpaths}] Once per battle round, instead
        of moving in your movement phase, one unit from this battalion that is
        more than 3~inch from any enemy models can move along the shadowpaths.
        If it does so, remove the unit from the battlefield, then set it up
        anywhere on the battlefield more than 9~inch from any enemy models.
        This is its move for that movement phase.
\end{description}

\subsection{\textsc{Shadowhammer Compact}}
Requires the following organisation of units:
\begin{itemize}
    \tightlist
    \item Slaughter Queen
    \item 2 units of Witch Aelves
    \item 1 unit of Khinarai Harpies
    \item 1 unit of Blood Stalkers
    \item 2 units if Liberators
    \item 1 unit of Prosecutors
    \item 1 unit of Judicators
\end{itemize}
\begin{description}
    \item [\htarget{righteous-fervour}{Righteous Fervour}] In your hero phase,
        choose one \textsc{DoK} unit from this battalion and one STORMCAST
        ETERNAL unit from this battalion that are within 6~inch of each other.
        Both units can either make a normal move as if it were your movement
        phase, shoot as if it were your shooting phase, or pile in and attack
        as if it were the combat phase. Both units must perform the same
        action.
\end{description}

\newpage
\htarget{library}{\section{Units}}
\htarget{morathi}{\subsection{\textsc{Morathi}}}
\htarget{morathi-wizard}{\subsubsection{\textsc{Morathi, High Oracle of Khaine}}}

\stats{6}{4}{9}{6}
\statstable{
    \Melee & Heartrender        & 2 & 3 & 3+ & 3+ & -1 & 1\textsc{D}3 \\
    \Melee & Bladed Wings       & 2 & 6 & 3+ & 3+ & -1 & 1 \\
    }

\begin{description}
    \item [\htarget{monstrous-transformation}{Monstrous Transformation}] At the
        start of your hero phase, Morathi can transform into her monstrous
        aspect. See the Morathi, the Shadow Queen warscroll for
        a description of how Morathi transforms.
    \item [\htarget{the-truth-revealed}{The Truth Revealed}] If Morathi is
        wounded, there is a chance she will no longer be able to contain her
        wrath and will transform into her monstrous aspect. Roll a dice at the
        start of your hero phase. If the result is equal to or less than the
        number of wounds currently allocated to Morathi, she transforms as
        described on the Morathi, the Shadow Queen warscroll.
    \item [\htarget{the-iron-heart-of-khaine}{The Iron Heart of Khaine}]
        Morathi, High Oracle of Khaine cannot be healed, but no more than
        \marginpar{Not more than three wounds per round!}
        3 wounds can be allocated to her in any one turn. Any additional wounds
        and/or mortal wounds allocated to her in the same turn are negated and
        have no effect.
    \item [\htarget{sorceress-supreme}{Sorceress Supreme}] Add 1 to casting and
        unbinding rolls made for Morathi, High Oracle of Khaine. In addition,
        double the range of spells she attempts to cast.
    \item [\htarget{enchanting-beauty}{Enchanting Beauty}] Subtract 1 from the
        hit rolls of attacks that target Morathi, High Oracle of Khaine.
    \item [\htarget{spell:arnzipals-black-horror}{Arnzipal's Black Horror}]
        Arnzipal's Black Horror has a casting value of 7. If successfully cast,
        pick an enemy unit within 18~inch of the caster that is visible to them
        and roll a dice. On a 1 that unit suffers 1 mortal wound, on a 2 or
        3 it suffers D3 mortal wounds, and on a 4+ it suffers D6 mortal
        wounds.
    \item [\htarget{worship-through-bloodshed}{Worship Through Bloodshed}]
        Command Ability. If Morathi, High Oracle of Khaine is your general, you
        can use this ability. If you do, pick up to 2 friendly \textsc{DoK}
        units within 14~inch of Morathi. Those units can immediately shoot as
        if it were the shooting phase. Alternatively, if either unit is within
        3~inch of an enemy unit, it can instead be chosen to pile in and attack
        as if it were the combat phase.
\end{description}

Morathi can attempt to cast three spells in your hero phase, and attempt to
\marginpar{Three spells, Unbind two.} unbind two spells in the enemy hero
phase. She knows the \hyperref[arcane-bolt]{Arcane Bolt}, Mystic Shield and
Arnzipal's Black Horror spells.  She is adding +1 to her rolls and doubling the
range of her spells.  Her unique spell is
\hyperref[spell:arnzipals-black-horror]{Arnzipals Black Horror} which basically
smacks a unit with a random number mortal wounds.\\ Her 
\hyperref[worship-through-bloodshed]{command ability} allows her to pick two friendly
\textsc{DoK} units within 14~inch and let them make an immediate shooting
attack or allow them to pile in an make a melee attack.\\ \marginpar{Let others
shoot or pile in during hero phase} In close combat, she's a devil, throwing out nine
3+/3+/-1/1 damage or D3 damage attacks between her Heartrender and her Bladed
Wings (note she cannot fly though), all the while imposing a -1 to hit when
enemies attempt to swing back at her and she can only take a maximum of three
wounds per turn. 

Morathi \emph{loves} the Khailebron Temple and the Khailebron Temple loves her.
If you don't transform her at the right time, you risk losing out on her
contribution to combat in the late game.
\hyperref[temple:khailebron]{Khailebron} solves that by teleporting her
\marginpar{Teleport her into battle: Needs a general near by.}
wherever she needs to be to rip things to shreds, in addition to making her
nearly impossible to shoot before that time comes. Let her hang out in the
backfield (next to your General, i.\,e.\, do not make her your General!),
flexing that 36~inch Sorceress Supreme magical range to snipe enemy heroes or
buff your own units, then morph her into Shadow Queen mode and pop her up right
behind whatever is left of your enemy's lines.\\
\htarget{morathi-the-shadow-queen}{\subsubsection{\textsc{Morathi, the Shadow Queen}}}
\stats{*}{4}{9}{12}
\statstable{
    \Missile & Gaze of Morathi & 6 & 1 & 2 & \multicolumn{3}{c}{see below}\\[1ex]
    \Melee & Heartrender         & 2 & * & 3 & 3 & -2 & 3 \\
    \Melee & Crown of Serpents   & 1 & 5 & 3 & 3 & -2 & 1 \\
    \Melee & Envenomed Tail      & 3 & 1 & 3 & 3 & -2 & * \\
    }


Morathi, the Shadow Queen is a \textsc{Wizard}. She can attempt to cast one
spell in your hero phase, and attempt to unbind one spell in the enemy hero
phase. She knows the Arcane Bolt, Mystic Shield and Arnzipal’s Black Horror
spells.

\begin{description}
   \item [\htarget{gaze-of-morathi}{Gaze of Morathi}] If a target is hit by the
        Gaze of Morathi, pick a model in the target unit and roll a dice. If the
        result exceeds that model's Wounds characteristic, it is slain.
\end{description}

\begin{scriptsize}
\noindent
    \begin{tabular}{cccc} %8 cols
    \toprule
    Wounds Suffered & Move & Heartrender & Envenomed Tail \\
    \midrule
    0--2 & 14 &  6  & 6 \\
    3--4 & 12 &  5  & D6 \\
    5--7 & 10 &  4  & D6 \\
    0--2 &  8 &  3  & D3 \\
    0--2 &  6 &  2  & D3 \\
    \bottomrule
\end{tabular}
\end{scriptsize}


\newpage %------------------------------------------------------------------------
\htarget{hag-queen}{\subsection{\textsc{Hag Queen}}}
\stats{6}{5}{8}{5}
\statstable{
    \Melee & Blade of Khaine    & 1 & 4 & 3 & 4 & -1 & 1 \\[1ex]
    }

\begin{description}
    \item [Priestess of Khaine] In your \hyperref[hero-phase]{hero phase},
        a Hag Queen can pray once. If she does, pick a prayer she knows and
        roll a dice. On a result of 1 she is found unworthy and suffers
        1 mortal wound. On a 2 nothing happens. On a 3+ the prayer is
        successful and its effect takes place.  A Hag Queen knows the Rune of
        Khaine and Touch of Death prayers:
        \begin{description}
            \item [{\hyperref[rune-of-khaine]{Rune of Khaine}}] The Hag Queen's
                Blade of Khaine has a Damage characteristic of D3 instead of
                1 until your next hero phase.
            \item [{\hyperref[touch-of-death]{Touch of Death}}] Pick a unit within
                3~inch of this model and then hide a dice in one of your hands.
                Your opponent must pick a hand; if that hand is holding the
                dice, the unit you picked suffers D3 mortal wounds.
        \end{description}
    \item [\htarget{witchbrew}{Witchbrew}] In your \hyperref[hero-phase]{hero
        phase}, you can pick a friendly \textsc{DoK} unit within 3~inch of this model to
        drink witchbrew. If you do,\marginpar{re-roll failed to wound rolls!
        and no battleshock test} then until your next hero phase you can
        re-roll failed wound rolls for that unit's melee weapons, and you do
        not need to take battleshock tests for the unit.
\end{description}


\newpage %------------------------------------------------------------------------
\htarget{slaughter-queen}{\subsection{\textsc{Slaughter Queen}}}
\stats{6}{5}{9}{5}
\statstable{
    \Melee & Blade of Khaine    & 1 & 4 & 3+ & 4+ & -1 & 1 \\
    \Melee & Deathsword         & 1 & 3 & 3+ & 3+ & -1 & D3\\
    }

\begin{description}
    \item [Priestess of Khaine] In your \hyperref[hero-phase]{hero phase},
        a Hag Queen can pray once. If she does, pick a prayer she knows and
        roll a dice. On a result of 1 she is found unworthy and suffers
        1 mortal wound. On a 2 nothing happens. On a 3+ the prayer is
          successful and its effect takes place.  A Hag Queen knows the Rune of
          Khaine and Touch of Death prayers:
        \begin{description}
            \item [{\hyperref[rune-of-khaine]{Rune of Khaine}}] The Hag Queen's
                Blade of Khaine has a Damage characteristic of D3 instead of
                1 until your next hero phase.
            \item [{\hyperref[touch-of-death]{Touch of Death}}] Pick a unit within
                3~inch of this model and then hide a dice in one of your hands.
                Your opponent must pick a hand; if that hand is holding the
                dice, the unit you picked suffers D3 mortal wounds.
            \item [\htarget{dance-of-doom}{Dance of Doom}] Until your next hero
                phase, this model can be chosen to pile in and attack twice in
                the combat phase.
        \end{description}
    \item [\htarget{pact-of-blood}{Pact of Blood}] A Slaughter Queen can
        attempt to unbind one spell in the enemy hero phase as if it were
        a \textsc{wizard}.
    \item [\htarget{orgy-of-slaughter}{Orgy of Slaughter}] If this model is
        your general, you can use this ability. If you do, pick a friendly
        DoK unit within 14~inch of this model. If that unit is
        within 3~inch of an enemy unit, it can pile in and attack as if it were the
        combat phase.
\end{description}

Primary priest, always casting on a 3+ but wounding herself on a 1.
Good buff increasing her meh knife's damage to D3, essentially matching her Deathsword,
which will put the lady at 7 damage D3 attacks, enough to kill bigger units all
on her own. Her other good prayer is the ability to fight twice in the combat
phase, which you should use if she goes up against hordes rather than monsters.
She can also have this ability as a command ability, allowing a friendly unit
to pile in during the hero phase. \\
Leave her at home unless she's your general or you're running Shadowhammer
compact; If you take her as general her command ability makes your Witch
Aelves kill more than what a Hag Queen would do, making her a better option in
this case. \\
One thing to remember is that priests get an extra prayer from the list of
khainite prayers, and therefore can actually pray twice in your hero phase.


\newpage %------------------------------------------------------------------------
\htarget{avatar-of-khaine}{\subsection{\textsc{Avatar of Khaine}}}
\stats{9}{4}{10}{9}
\statstable{
    \Melee & Blade of Khaine    & 1 & 4 & 3+ & 4+ & -1 & 1 \\
    \Melee & Deathsword         & 1 & 3 & 3+ & 3+ & -1 & D3 \\
    }

Needs to be animated. This becomes less of an issue if you have the Blood Rites
allegiance ability, which automatically animates your statue from turn three
onwards, but you want this thing moving before then because it's a heavy
hitter.\\ 
Buff your Avatar with Mindrazor and watch him wreck everything in
range. Thanks to his high bravery, his attacks will nearly always be four
attacks with -3 rend and 4 damage each. Bravery 11 (10+1 thanks to
\hyperref[idol-of-worship]{Idol of Worship}) will beat the bravery 10 of any of
these units.

\begin{description}
    \item [\htarget{wrath-of-khaine}{Wrath of Khaine}] If your army includes
        \marginpar{Remember that the Wrath of Khaine prayer is not one of the
        six Prayers of the Khainite cult, so can be attempted by more than one
        \textsc{Priest} in the same turn}
        any Avatars of Khaine, friendly \textsc{DoK} priests know the Wrath of Khaine
        \marginpar{Basically a prayer used to activate in rounds 1--3}
        prayer in addition to any other prayers they know:  Pick a friendly
        Avatar of Khaine on the battlefield – until your next hero phase, it is
        now Animated. \\
        \textbf{Animated:} The Avatar of Khaine cannot move, cannot shoot
        and cannot be selected to fight unless a friendly \textsc{DoK}
        priest used the Wrath of Khaine prayer to animate it in your preceding
        hero phase. Even if this model has not been animated it is still
        treated as a model in your army, with the exception that enemy units
        that begin their movement phase within 3~inch of it can either remain
        stationary or move normally -- they do not have to retreat unless there
        is another enemy unit within 3~inch of them.
    \item [{\hyperref[idol-of-worship]{Idol of Worship}}] Add 1 to the Bravery
        characteristic of friendly \textsc{DoK} units that are within 7~inch of any
        friendly Avatars of Khaine.
\end{description}

\newpage %------------------------------------------------------------------------
\htarget{cauldron}{\subsection{\textsc{Cauldron of Blood}}}
If you are considering which way to go between the Shrine or the
Cauldron. You are mostly weighing up whether you want the
\hyperref[aura-of-agony]{Aura of Agony} for causing mortal wounds in a radius,
or the \hyperref[bloodshield]{Bloodshield} for giving your sisters a +1 to
saves.\\

\begin{description}
    \item [\htarget{bladed-impact}{Bladed Impact}] Roll a dice if this model
        ends a charge move within 1~inch of any enemy units. On a 2+ the
        nearest enemy unit suffers D3 mortal wounds.
    \item [\htarget{bloodshield}{Bloodshield}] The powerful magic that fuels
        the Cauldron of Blood grants it and nearby followers protection. The
        range of this ability is shown in the damage table above. Add 1 to the
        saving throw of friendly \textsc{DoK} units that are wholly
        within this range of this model. A unit can only be affected by
        a single Cauldron of Blood’s Bloodshield ability at any one time.
\end{description}


\htarget{bloodwrack-shrine}{\subsection{\textsc{Bloodwrack Shrine}}}

\stats{13}{*}{5}{9}
\statstable{
    \Missile & Bloodwrack Stare                 & 10& 
        \multicolumn{5}{c}{\hyperref[bloodwrack-stare]{Bloodwrack Stare}}\\[1ex]
    \Melee   & Tail of Serpents                 & 2 & D6& 4+ & 4+ & - & 1\\
    \Melee   & Whisperclaw                      & 1& 4& 4+ & 3+ & - & 1\\
    \Melee   & Bloodwrack Speer                 & 2& 2& 3+ & 3+ & -1 & D3\\
    \Melee   & Shinekeepers' Goadstaves         & 2& *& 3+ & 3+ & - & D3\\
    }

\begin{scriptsize}
\noindent
    \begin{tabular}{cccc} %8 cols
    \toprule
    Wounds Suffered & Move & Goadstaves & Aura of Agony \\
    \midrule
    0--2 & 6 &  6  & 2+ \\
    3--5 & 5 &  5  & 2+ \\
    6--8 & 4 &  4  & 3+ \\
    9--10& 3 &  3  & 4+ \\
    11+  & 2 &  2  & 5+ \\
    \bottomrule
\end{tabular}
\end{scriptsize}
\vspace{1em}

While you should probably get a Cauldron of Blood first, a Bloodwreck Shrine is
only 80 points more than the Medusa on her own. It costs you 2~inches of
movement but it more than doubles your wounds, gives you the Shrinekeepers
attacks, the Bladed Impact rule and gives you the Aura of Agony, which causes
D3 mortal wounds all enemy units within a 7~inch radius on a diminishing die
roll, starting at 2+, which coupled with her Bloodwrack Stare means that the
Shrine has the potential to be doling out a lot of mortal wounds in a given
turn. They are also Wizards now, being able to cast/unbind a single spell per
turn.\\
\marginpar{Remember that a Bloodwrack Shrine can cast Steed of Shadows on
itself to fly 16~inch and get that Aura of Agony plus Bloodwrack Stare where it
would hurt the most.}

\begin{description}
    \item [{\hyperref[bladed-impact]{Bladed Impact}}]
    \item [{\htarget{bloodwrack-stare}{Bloodwrack Stare}}] When making
        a Bloodwrack Stare attack, pick a unit that is visible to the
        Bloodwrack Shrine and roll a dice for each model in that unit that is
        within range; for each roll of 5+ the unit suffers 1 mortal wound.
    \item [\htarget{aura-of-agony}{Aura of Agony}] Bloodwrack Shrines emit an
        aura that wracks enemies with waves of agony. Roll a dice for each
        enemy unit within 7~inch of any friendly Bloodwrack Shrines at the start of
        your hero phase. If the dice roll equals or beats the score listed on
        the damage table above, that unit suffers D3 mortal wounds as pure
        agony courses through them.
    \item [\htarget{spell:enfeebling-foe}{Enfeebling Foe}] Enfeebling Foe has
        a casting value of 5. If successfully cast, pick a unit within 18~inch of
        the caster that is visible to them. Until your next hero phase,
        subtract 1 from wound rolls for that unit in the combat phase.
\end{description}


\htarget{slaughter-queen-on-cauldron-of-blood}{%
    \subsection{\textsc{Slaughter Queen on Cauldron of Blood}}}


\stats{13}{*}{5}{9}
\statstable{
    \Missile & Torrent of Burning Blood         & 10 & 6 & 3+ & 3+ & -1 & 1\\[1ex]
    \Melee   & Witch Aelves' Sacrificial Knives & 1  & * & 3+ & 4+ & - & 1\\
    \Melee   & Slaughter Queen's Blade of Khaine& 1  & 4 & 3+ & 4+ & -1 & 1\\
    \Melee   & Slaughter Queen's Deathsword     & 1  & 3 & 3+ & 3+ & -1 & D3\\
    \Melee   & Avatar of Khaine's Sword         & 2  & 4 & 3+ & 3+ & -2 & 3\\
    }

\begin{scriptsize}
\noindent
    \begin{tabular}{cccc} %8 cols
    \toprule
    Wounds Suffered & Move & Sacrificial Knives & Bloodshield \\
    \midrule
    0--2 & 6 &  8  & 18 \\
    3--5 & 5 &  7  & 14 \\
    6--8 & 4 &  6  & 10 \\
    9--10& 3 &  5  &  6 \\
    11+  & 2 &  4  &  2 \\
    \bottomrule
\end{tabular}
\end{scriptsize}
\vspace{1em}


\begin{description}
    \item [{\hyperref[bladed-impact]{Bladed Impact}}] 
    \item [{\hyperref[bloodshield]{Bloodshield}}]
    \item [{\hyperref[wrath-of-khaine]{Wrath of Khaine}}] 
    \item [{\hyperref[idol-of-worship]{Idol of Worship}}] Add 1 to the Bravery
        characteristic of friendly \textsc{DoK} units that are within 7~inch
        of any friendly Avatars of Khaine.
    \item [{\hyperref[pact-of-blood]{Pact of Blood}}] A Slaughter Queen on
        a Cauldron of Blood can attempt to unbind one spell in the enemy hero
        phase as if it were a \textsc{wizard}.
    \item [Command Ability: {\hyperref[orgy-of-slaughter]{Orgy of Slaughter}}] 
    \item [Priestess of Khaine] In your \hyperref[hero-phase]{hero phase},
        a Hag Queen can pray once. If she does, pick a prayer she knows and
        roll a dice. On a result of 1 she is found unworthy and suffers
        1 mortal wound. On a 2 nothing happens. On a 3+ the prayer is
        successful and its effect takes place.  A Hag Queen knows the Rune of
        Khaine and Touch of Death prayers:
        \begin{description}
            \item [{\hyperref[rune-of-khaine]{Rune of Khaine}}] The Hag Queen's
                Blade of Khaine has a Damage characteristic of D3 instead of
                1 until your next hero phase.
            \item [{\hyperref[touch-of-death]{Touch of Death}}] Pick a unit within
                3~inch of this model and then hide a dice in one of your hands.
                Your opponent must pick a hand; if that hand is holding the
                dice, the unit you picked suffers D3 mortal wounds.
        \end{description}
\end{description}

Absolutely essential for turning an assortment of powerful units into
a coherent army.  Basically a straight upgrade for one of the Queens with an
Avatar of Khaine strapped to the same model It also has extra attacks thanks to
the attendant sisters and an added charge bonus which inflicts Mortal Wounds.
It also allows \textsc{DoK} units within a diminishing range to add +1 to their
save rolls, definitely worth having considering how fragile the army is in
general.\\

If you want a rapetrain, make a Slaughter Queen on this thing
your general, then give her \hyperref[mistress-of-poisons]{Mistress of
Poisons}, the \hyperref[crone-blade]{Crone Blade} and
\hyperref[catechism-of-murder]{Catechism of Murder}, wait for Turn~3 then
charge something: you'll get +1 to the damage of all your melee hits since
\hyperref[mistress-of-poisons]{Mistress of Poisons} works for all of the
cauldron's attacks, and with Crone Blade on the Avatar's attack you get
more than a chance to cause casualties and recover wounds, since it now
does 4 damages per non saved attack, compensating for your fragility (you
still have 13 wounds with a 4+ save but some generals can be tougher);
factor in \hyperref[catechism-of-murder]{Catechism of Murder} and a self
cast \hyperref[orgy-of-slaughter]{Orgy of Slaughter}. \emph{If} you don't
care about the Crone Blade, take a \hyperref[hagbrew]{Hagbrew} instead.\\


\htarget{hag-queen-on-cauldron}{\subsection{\textsc{Hag Queen on Cauldron of Blood}}}
\stats{*}{5}{8}{13}
\statstable{
    \Missile & Torrent of Burning Blood     & 10& 6 & 3+ & 3+ & -1 & 1 \\[1ex]
    \Melee & With Aelves Sacrificial Knives & 1 & * & 3+ & 4+ & -  & 1 \\
    \Melee & Hag Queen's Blade of Khaine    & 1 & 4 & 3+ & 4+ & -1 & 1 \\
    \Melee & Avatar of Khaines Sword & 2 & 4 & 3+ & 3+ & -2 & 3 \\
    }

\begin{description}
    \item [{\hyperref[bladed-impact]{Bladed Impact}}] 
    \item [{\hyperref[bloodshield]{Bloodshield}}]
    \item [{\hyperref[wrath-of-khaine]{Wrath of Khaine}}] 
    \item [{\hyperref[witchbrew]{Witchbrew}}]
    \item [{\hyperref[idol-of-worship]{Idol of Worship}}] Add 1 to the Bravery
        characteristic of friendly \textsc{DoK} units that are within 7~inch
        of any friendly AVATARS OF KHAINE.
    \item [Priestess of Khaine] In your \hyperref[hero-phase]{hero phase},
        a Hag Queen can pray once. If she does, pick a prayer she knows and
        roll a dice. On a result of 1 she is found unworthy and suffers
        1 mortal wound. On a 2 nothing happens. On a 3+ the prayer is
        successful and its effect takes place.  A Hag Queen knows the Rune of
        Khaine and Touch of Death prayers:
        \begin{description}
            \item [{\hyperref[rune-of-khaine]{Rune of Khaine}}] The Hag Queen's
                Blade of Khaine has a Damage characteristic of D3 instead of
                1 until your next hero phase.
            \item [{\hyperref[touch-of-death]{Touch of Death}}] Pick a unit within
                3~inch of this model and then hide a dice in one of your hands.
                Your opponent must pick a hand; if that hand is holding the
                dice, the unit you picked suffers D3 mortal wounds.
        \end{description}
\end{description}


\newpage %------------------------------------------------------------------------
\htarget{witch-aelves}{\subsection{\textsc{Witch Aelves}}}
\stats{6}{6}{7}{1}
\statstable{
    \Melee & Sacrificial Knives & 1 & 2 & 3+ & 4+ & - & 1 \\
    }

\textsc{Battleline} If you take the Buckler
    \marginpar{Always make saving throws with bucklers.\\ 
    Always take a big block of 30 Witch Aelves! By stacking bonuses, you can
    put out a massive number of attacks} 
then they get a 5+ save with any roll of a six causing an immediate mortal
wound on the attacker. If you'd rather have them be more killy, you can swap
the buckler for an extra sacrificial knife which grants them another attack, up
to three per model, which makes them rather choppy for a cheap unit, this can
be further upgraded thanks to the
\hyperref[frenzied-fervour]{Frenzied Fervour} rule which grants them another
extra attack whenever they are within 8~inch of \textsc{DoK} Hero in the combat phase,
with each potentially throwing out four 3+/4+/-/1 attacks per turn with
rerolls depending on what turn it is.

\begin{description}
    \item [\htarget{frenzied-fervour}{Frenzied Fervour}] If this unit is within
        \marginpar{Keep them close to a hero. Heroes are:
            \hyperref[slaughter-queen]{Slaughter Queen} 
            \hyperref[slaughter-queen-on-cauldron-of-blood]{(on Cauldron)}, 
            \hyperref[bloodwrack-medusa]{Bloodwrack Medusa}, 
            \hyperref[morathi-wizard]{Morathi}, 
            \hyperref[morathi-the-shadow-queen]{Morathi, Shadow Queen}, 
            \hyperref[hag-queen]{Hag Queen},
            \hyperref[hag-queen-on-cauldron]{(on Cauldron)},
            \hyperref[slaughter-queen]{Slaughter Queen}
            } 
        8~inch of any friendly \textsc{DoK} heroes in the combat phase, add 1 to the
        Attacks characteristic of its Sacrificial Knives until the end of the
        phase. 
    \item [Hornblower] Models in this unit can be Hornblowers. A unit that
        includes any Hornblowers can charge even if it ran in the same turn.
    \item [Standard Bearer] Models in this unit can be Standard Bearers. If
        a unit includes any Standard Bearers when you take a battleshock test
        for it, roll two dice instead of one and discard the highest result.
\end{description}

\newpage %------------------------------------------------------------------------
\htarget{sisters-of-slaughter}{\subsection{\textsc{Sisters of Slaughter}}}
\stats{6}{6}{7}{1}
\statstable{
    \Melee & Barbed Whip        & 2 & 2 & 3+ & 4+ & - & 1 \\
    \Melee & Sacrificial Knives & 1 & 1 & 3+ & 4+ & - & 1 \\[1ex]
    }

\textsc{Battleline} --- Their unique special rule is
that they can be chosen to pile in and fight up from up to 6~inch away rather than
3~inch, which means they have a massive threat radius without even charging and can
jump on to the enemy during his own Combat Phase. A good use of these ladies
might be to hover around your own units and perform a counter-charge whenever
one of your weaker units gets threatened, dog-piling on the enemy when he
thought he had the advantage. 
\begin{description}
    \item [Hornblower] Models in this unit can be Hornblowers. A unit that
        includes any Hornblowers can charge even if it ran in the same turn.
    \item [Standard Bearer] Models in this unit can be Standard Bearers. If
        a unit includes any Standard Bearers when you take a battleshock test
        for it, roll two dice instead of one and discard the highest result.
    \item [Dance of Death] Sisters of Slaughter can be chosen to pile in and
        attack in the combat phase if they are within 6~inch of an enemy, and
        can move up to 6~inch when they pile in.
\end{description}

\newpage %------------------------------------------------------------------------
\htarget{blood-sisters}{\subsection{\textsc{Blood Sisters}}}
\stats{8}{5}{8}{2}
\statstable{
    \Melee   & Heartshard Glaive    & 2 & 3 & 3+ & 3+ & -1& 1 \\
    \Melee   & Crystal Touch        & 1 & 1 & 4+ & 
        \multicolumn{3}{c}{\hyperref[turned-to-crystal]{Turned to Crystal}}\\
    }

\textsc{Battleline} if your general is
a \hyperref[bloodwrack-medusa]{Bloodwrack Medusa}. Your gorgon snake-chicks
whose job is to be the army's elite melee anchor unit. Armed with Glaives and
their Crystal Touch attacks, they can throw out a reliable amount of damage
every turn. Glaives have three 3+/3+/-1/1 attacks at 2~inch range which would
be decent before you consider rerolls from Blood Rites or further buffs from
prayers.\\
The Crystal Touch is a kicker though, causing an instant mortal wound on every
hit. Making Blood Sisters the ones you want to throw at the enemy elite to get
maximum advantage of the Crystal Touch. The main problem is that they cost
quite a lot for only a few models, yes they have two-wounds each but they are
still fragile with a 5+ save (unless you bring a cauldron, which you should).
Large units are also a must simply because of the massive discount they get in
groups of 20. \\

\begin{description}
    \item  [\htarget{turned-to-crystal}{Turned to Crystal}] Each time you
        score a hit with a Crystal Touch, the target suffers 1 mortal wound.
\end{description}

Even 5 snake ladies are a strong close combat unit
\marginpar{Consider \hyperref[draichi-ganeth]{Draichi Ganeth’s} charge bonus
for mortal wounds on 5+ rather than~6.} 
but their damage output can be increased to ridicules levels. 10 snakes with
\hyperref[catechism-of-murder]{catechism of murder}, mindrazor and witch brew
in a \hyperref[hagg-nar]{Hagg Nar} list will have 30 -2 rend and 2 Damage
attacks, 10 crystal touch attacks, all with re-rolls to-hit and to-wound plus
exploding sixes.\\
Don't forget the defensive buffs, with a Cauldron, \hyperref[hagg-nar]{Hagg
Narr}, and the reroll prayer you can make a big unit of the snake ladies 4+ (3+
if you can find a nice big terrain piece) 5++, 5+++ which makes them actually
ridiculously tough.  


\newpage %------------------------------------------------------------------------
\htarget{blood-stalkers}{\subsection{\textsc{Blood Stalkers}}}
\stats{8}{5}{8}{2}
\statstable{
    \Missile & Heartseeker Bow                  & 24& 1 & 3+ & 3+ & -1 & 1 \\[1ex]
    \Melee   & Sacrificial Dagger               & 1 & 2 & 3+ & 4+ & - & 1 \\
    \Melee   & Blood Wyrm (Leader)              & 1 & 1 & 3+ & 3+ & - & 1 \\
    }

The ranged snake ladies. Yes you also have Khinerai Heartrenders now, but the
Stalkers have the longer range and fewer rules for moving about, making them
your primary fire support unit rather than the mobile harassing unit of
harpies. Rolling 6+ to hit makes the bow do a mortal wound instead of the
normal damage, which is nice to have but shouldn't be relied upon since you
can't do massed ranged firepower like some other armies can. They are only
average melee combatants.\\ 

\begin{description}
    \item [\htarget{heartseeker}{Heartseekers}] Each time you make a hit roll
        of 6+ for this unit in the shooting phase, the target suffers 1 mortal
        wound instead of the normal damage.
\end{description}

\htarget{bloodwrack-medusa}{\subsection{\textsc{Bloodwrack Medusa}}}
\stats{8}{5}{8}{6}
\statstable{
    \Missile & Bloodwrack Stare                 & 10& 
        \multicolumn{5}{c}{\hyperref[bloodwrack-stare]{Bloodwrack Stare}}\\[1ex]
    \Melee   & Whisperclaw                      & 1& 4& 3+ & 3+ & - & 1\\
    \Melee   & Tail of Serpents                 & 2 & D6& 4+ & 4+ & - & 1\\
    \Melee   & Bloodwrack Speer                 & 2& 2& 3+ & 3+ & -1 & D3\\
    }

\begin{description}
    \item [{\hyperref[bloodwrack-stare]{Bloodwrack Stare}}] When making
        a Bloodwrack Stare attack, pick a unit that is visible to the
        Bloodwrack Medusa and roll a dice for \emph{each model} in that unit
        that is within range; for each roll of 5+ the unit suffers 1 mortal
        wound.
\end{description}

Your generic Wizard, capable of tossing off a spell that makes it harder for
the enemy to wound you, which is also nice, plus Mindrazor or whatever you
like. Also, you know how Wizards tend to be useless in close combat and
shooting? Well this is Daughters of Khaine and everyone has to pull their
weight. Has an absolutely brutal shooting attack, where you roll a dice for
every model in a unit and every 5+ is a Mortal Wound, and has a metric ton of
close combat attacks, with a theoretical maximum of 16 wounds (average dice
rolls will probably render it closer to 5 or 6, depending on what turn it is).
You already want her slinging spells left and right, and she can easily be used
to quickly clear out a unit of archers or something while your more important
units take out tougher enemies. Definitely needs to be on your list.

A Bloodwrack Medusa can attempt to cast one spell in your hero phase, and
attempt to unbind one spell in the enemy hero phase. It knows the
\hyperref[arcane-bolt]{Arcane Bolt}, \hyperref[mystic-shield]{Mystic Shield}
and \hyperref[spell:enfeebling-foe]{Enfeebling Foe} spells.\\



\newpage %------------------------------------------------------------------------
\htarget{doomfire-warlocks}{\subsection{\textsc{Doomfire Warlocks}}}
\stats{14}{5}{6}{2}
\statstable{
    \Missile & Doomfire Crossbow                & 10& 2& 4+ & 4+ & - & 1\\[1ex]
    \Melee   & Cursed Scimitar                  & 1& 2& 4+ & 4+ & -1 & 1\\
    \Melee   & Dark Steed's Vicious Bite        & 1& 2& 4+ & 5+ & - & 1\\
    }

\begin{description}
    \item [\htarget{doomfire-coven}{Doomfire Coven}] Add 1 to casting and
        unbinding rolls for this unit if it has 10 or more models.
    \item [\htarget{doomfire}{Doomfire}] Doomfire has a casting value of 6. If
        successfully cast, pick an enemy unit within 18~inch of any model in
        the casting unit that is visible to it. The target unit suffers D3
        mortal wounds if the casting unit has fewer than 5 models, D6 mortal
        wounds if it has 5 to 9 models, or 6 mortal wounds if it has 10 or more
        models.
\end{description}

They are excessively fast at 14~inch and they're pretty tough unit of
\textsc{wizards}. This makes them incredibly versatile already, letting you
play small units to either shoot Mortal Wounds or buff your squishy ladies, but
the really great part is their unique spell. First, it's casting value 5, so
very easy to get off, then you get +1 to cast if the unit has 10+ models and
then its a damage spell that scales to your unit size, topping at a flat
6 Mortal Wounds at 10+ models. Six. Mortal. Wounds. 
\begin{itemize}
    \item Another way to use them is to ignore their trademark spell and see
        them as a mobile buff support unit. Most time than not \textsc{DoK}
        tends to be a single directed steam roller with not many units to sit
        back and cap objectives. This is where The warlocks comes in, they are
        fast enough to cap and tough enough to stick a little longer vs smaller
        opposing units compared to using harpies for the similar role esp if
        they are within range to the blood shield! To top it off their insane
        14~inch movement allows you to position your warlocks in such a way
        what they can be outside the 30~inch dispel radius from your opponent
        and still in range to buff your own units 
    \item Even though they are \textsc{Cavalry}, there are multiple ways of
        being able to teleport them into the face of your opponent on turn 1.
        Taking a \hyperref[shadowpaths]{Shadow Patrol} with two units of 10
        Warlocks and a General with the Khailebron Command Trait
        \hyperref[mistress-of-illusion]{Mistress of Illusion} can do just this.
        That is 6 Mortal Wounds plus 40 4+/4+/-/1 Crossbow shots, and whatever
        spells you loaded onto the second unit hitting your opponent at once.
\end{itemize}
A unit of Doomfire Warlocks can attempt to cast one spell in your hero phase,
and attempt to unbind one spell in the enemy hero phase. A unit of Doomfire
Warlocks knows the \hyperref[arcane-bolt]{Arcane Bolt},
\hyperref[mystic-shield]{Mystic Shield} and Doomfire spells.


\newpage %------------------------------------------------------------------------
\htarget{khinerai-lifetakers}{\subsection{\textsc{Khinerai Lifetakers}}}
\stats{14}{6}{7}{1}
\statstable{
    \Melee   & Barbed Sickle  & 1& 2& 3+ & 4+ & - & 1\\
    }

\begin{description}
    \item [Descend to Battle] Instead of setting
        up this unit on the battlefield, you can place it to one side and say
        it is circling high above. In any of your movement phases, it can
        descend to battle – set up the unit anywhere on the battlefield that is
        more than 9~inch from any enemy models. This is their move for that
        movement phase.
    \item [\htarget{fight-and-flight}{Fight and Flight}] In the combat phase,
        after this unit has finished making all of its attacks, roll a dice: on
        a 4+ it can make a 6~inch normal move as if it were your movement
        phase, but it cannot run as part of this move.
    \item [\htarget{death-on-the-wind}{Death on the Wind}] 
        Add 1 to the Damage characteristic of this unit’s
        \marginpar{+1 damage when charging} 
        Barbed Sickles if it made a charge move in the same turn.
    \item [Heartpiercer Shield] In the combat
        phase, Khinerai Lifetakers have a Save characteristic of 5+. In
        addition, each time you make a save roll of 6 for such a unit in the
        combat phase (after re-rolls, but before any modifiers are applied),
        a Khinerai Lifetaker pierces her assailant’s heart with her shield
        -- the attacking unit suffers 1 mortal wound after it has made all of
        its attacks.
\end{description}

Lifetakers behave pretty much like flying Witch Aelves. They gain +1 damage
whenever they charge into combat and can also fall back from combat after they
have completed all of their attacks on a die roll of 4+, so you can escape
during an enemy turn and just charge them again on your own. Get enough of
these together and they'll be taking anything down.

\newpage %------------------------------------------------------------------------
\htarget{khinerai-heartrenders}{\subsection{\textsc{Khinerai Heartrenders}}}
\stats{14}{6}{7}{1}
\statstable{
    \Missile & Barbed Javelin                   & 12& 1& 3+ & 3+ & -1 & 1\\[1ex]
    \Melee   & Barbed Javelin                   & 2& 1& 4+ & 4+ & -1 & 1\\
    }

\begin{description}
    \item [\htarget{descend-to-battle}{Descend to Battle}] Instead of setting
        up this unit on the battlefield, you can place it to one side and say
        it is circling high above. In any of your movement phases, it can
        descend to battle – set up the unit anywhere on the battlefield that is
        more than 9~inch from any enemy models. This is their move for that
        movement phase.
    \item [\htarget{fire-and-flight}{Fire and Flight}] In your shooting phase,
        after this unit has finished making all of its attacks, roll a dice: on
        a 4+ it can make a 6~inch normal move as if it were your movement
        phase, but it cannot retreat or run as part of this move.
    \item [\htarget{death-from-above}{Death From Above}] This unit can shoot
        even it ran in the same turn.  In addition, in the shooting phase,
        change the Rend characteristic of this unit’s Barbed Javelins to -2 if
        it was set up on the battlefield in the same turn. 
    \item [\htarget{heartpiercer-shield}{Heartpiercer Shield}] In the combat
        phase, Khinerai Heartrenders have a Save characteristic of 5+. In
        addition, each time you make a save roll of 6 for such a unit in the
        combat phase (after re-rolls, but before any modifiers are applied),
        a Khinerai Heartrender pierces her assailant’s heart with her shield
        – the attacking unit suffers 1 mortal wound after it has made all of
        its attacks.
\end{description}

\marginpar{Drop two \hyperref[draichi-ganeth]{Draichi Ganeth} Slaughter Troupe
5-Harpy (each with an Allied Assassin) units in your opponents back field and
pepper a hiding hero with your Javelins until you can aggro one of his units to
come take care of you. Note: Be mindful of your surroundings and don't drop
into a bunch of gun lines\ldots but if you do, they'll spend a turn shooting
two 80 point units rather than your 300 point blobs of Witch's and Sister's.
Anyways, if you get something to bite, let them charge you, then
fly/run/retreat out of there towards your opponents now-isolated Hero, drop
your Allied Assassin's on him/her and decide where to go from there.}

The other of your two new Khinerai Harpy units. The Heartrender variant are
your ranged specialists who can fling their barbed javelins up to 12~inch doing
decent 3+/3+/-1/1 damage; although that's only one attack each.  They can Deep
Strike, arriving anywhere on the table more than 9~inch from the enemy when you
feel like, which is still happily within their shooting range, which also has
their Rend characteristic boosed to -2 for that alpha strike attack.  They can
also shoot after running and fall back after shooting on a die roll of 4+,
meaning that although 12~inch is quite a short range for a ranged weapon you
have a chance of darting in and out an inevitable counter-charge when the enemy
gets annoyed with you. They also have the Heartpiercer Shield which makes them
more resilient though admittedly you don't want this unit in melee because they
do so much better as a harassing unit. \\

Heartrenders can also serve as line backers, able to move and support your
units where needed. Use this when faced against armies that put out more
damage than they can handle, sitting them behind your hordes of
Witches/Sisters to take the brunt of your opponents abuse and chucking
Javelins at them then moving into combat to mop up whatever is left. \\





\section{References}
\subsection{Corerules}
\begin{description}
    \item [\htarget{corerule:at-the-double}{At the Double}] You can use this
        command ability after you make a run roll for a friendly unit that is
        within 6~inch of a friendly Hero, or
        12~inch of a friendly Hero that is a general.  If you do so, the run
        roll is treated as being a 6.  
    \item [\htarget{corerule:forward-to-victory}{Forward to Victory}] You can use
        this command ability after you make a charge roll for a friendly unit
        that is within 6~inch of a friendly Hero, or 12~inch of a friendly Hero
        that is a general.  If you do so, re-roll the charge roll.  
    \item [\htarget{corerule:inspiring-presence}{Inspiring Presence}] You can use
        this command ability at the start of the battleshock phase. If you do
        so, pick a friendly unit that is within 6~inch of friendly Hero, or 12~
        inch  of a friendly Hero that is a general. That unit does not have to
        take battleshock tests in that phase.
    \item [\htarget{corerule:look-out-sir}{LOOK OUT, SIR!}] You must subtract
        1 from hit rolls made for missile weapons if the target of the attack
        is an enemy Hero that is within 3~inch of an enemy unit that has 3 or
        more models. The Look Out, Sir! rule does not apply if the target Hero
        is a Monster.
    \item [\htarget{mystic-shield} {Mystic Shield}] has a casting value of 6.
        If successfully cast, pick the caster, or a friendly unit within 18
        inch of the caster and which is visible to them. You can add 1 to all
        save rolls for the unit you pick until the start of your next hero
        phase. 
    \item [\htarget{arcane-bolt}{Arcane Bolt}] Arcane Bolt has a casting value
        of 5. If successfully cast, pick an enemy unit within 18 inch of the
        caster and which is visible to them. The unit you pick suffer D3 mortal
        wounds. 
\end{description}

\subsection{Wizard-Spells}
\begin{description}
    \item [Rune of Khaine] The Hag Queen's Blade of
      Khaine has a Damage characteristic of D3 instead of
      1 until your next hero phase.
  \item [\htarget{touch-of-death}{Touch of Death}] Pick a unit within
      3~inch of this model and then hide a dice in one of your hands.  Your
      opponent must pick a hand; if that hand is holding the dice, the unit you
      picked suffers D3 mortal wounds.
    \item [\htarget{idol-of-worship}{Idol of Worship}] Add 1 to the Bravery
        characteristic of friendly \textsc{DoK} units that are within 7~inch
        of any friendly AVATARS OF KHAINE.
\end{description}
\subsection{Priest-Spells}


\subsection{Corerules}
\subsubsection{Command Traits}
Only your general has this, but you can choose what is cheesiest/fluffiest/Your
dudes-est or roll. Beware that if you have to choose a new general for whatever
reason, you have to roll. 
\begin{description}
    \item{Strategic Genius} At the start of the first battle round, you
        receive 1 extra command point. strong options for first turn
        shenanigans with 2 command points at your disposal.
    \item{Inspiring} Friendly ORDER units at 6 inch doesn't roll battleshock. If
        you expect heavy casualties, this trait is great, as it doesn't make
        a potentially bad situation worse. But be wary of becoming
        a DISTRACTION CARNIFEX and shot to death.
    \item{Reckless} Reroll charge of the general. If you have a melee
        powerhouse of a general (and Order has a few candidates), go ahead
    \item{Tenacious} +1 wound. Nice, but don’t get overconfident. A squishy
        wizard is still a squishy wizard, either with 5 or 6 wounds.
    \item{Legendary Fighter} +1 attack to a chosen melee weapon each combat
        phase. Depending on the hero, this can be shit or awesome.
    \item{Defense Master} Ignore wound and mortal wounds on 6+. Somewhat
        disappointing, but an additional save always is nice.
\end{description}

\subsubsection{Artifacts}
A HERO in your army can have one of these for free, and for each Batallion, you
include you can choose an additional HERO to carry another one.

\begin{description}
\item{Quicksilver Potion} Once per battle, you and your mount can attack
    before anyone, even in your opponent's turn. This can be potentially nasty
        in the right hands. Sure, a Freeguild General can't do much (you'll be
        surprised), but it's another story in a Lord Celestant on Dracoth or
        Stardrake or a Dreadlord. A good trick is to deliberately put them in
        charge distance from a nasty but manageable unit and drink it in the
        enemy's combat phase to take them for surprise and crush them.
\item{Obstinate Blade} -1 Rend to one of your melee weapons. Always useful. If
    you're not sure which artifact to take, this one is always a good option.
        Remember that a Rend -4 is usually overkill with the average saves out
        there, a -2/3 is way better if you have around 4~5 attacks with the
        said weapon.
\item{Relic Blade} +1 Damage to one of your melee weapons. As with the
    Obstinate blade, useful. Prioritize this if you know that your hero is
        going to face many multi-wound models, like cavalry or elite infantry.
\item{Hoarfrost} Choose one of your melee weapons, anyone who is wounded but
    survives, suffers -1 to Hit for the rest of the battle. This can be
        situational, but characters who rely on a flurry of attacks but are
        almost naked, like the Assasin or the Death Hag can be a lifesaver.
\item{Talisman of Blinding Light} Once per battle at the start of the combat
    phase, -1 to Hit to all who targets its bearer. Don't. There are better
        artifacts that do almost the same that last longer, like Hoarfrost, or
        better suited for defense, like the Phoenix Stone. And that's if you
        want to go defensive.
\item{Phoenix Stone} Heal 1 wound each friendly hero phase. Really good, even
    if somewhat underwhelming. That wound saved can be the difference between
        alive or dead, or a holy deliverer of the Pantheon's fury or a clawless
        newborn kitty. Combine with abilities or spells which heals like Lord
        Relictor's or Collegiate Life Battlemage's to great effect.
\end{description}

\section{Strategies and Tactics}
\begin{itemize}
    \item Use \hyperref[fanatical-faith]{Fanatical Faith} and \hyperref[martyrs-sacrifice]{Martyr's Sacrifice} on units that engage battle. If equipped with shields each unit that gets removed from the board potentially deals up to two mortal wounds.
\end{itemize}

\subsection{Deployment}
If you have fewer drops than your opponent, you will always be able to decide
who takes the first turn. With most games, the first turn is a blessing. Not so
much in Age of Sigmar.

The player that takes the second turn will have a chance to get a double turn.
However, optimize the gameplan that you can get a double turn on round three as
blood rites start to kick in then.

If it is not up to you to decide who takes the first turn, you can try to force
the hand of your opponent with your set-up. Do this by deploying aggressive,
forcing them to take the first turn or suffer heavy casualties and lose
objectives. Make sure you do this without overextending yourself.

\subsection{Threat range} The most important thing to consider when deploying
is how far all units can charge and shoot after one movement phase. This is
called the threat range.

\subsection{Charge} By destroying units on the alpha strike after charging, you
can deal massive damage each turn without having the strength of the unit
diminished.

All units you activate after the alpha strike will suffer one or more enemy
activations.  there are only four reasons to charge with a unit.

Units that have charged are always allowed to pile in 3 inch even if there is
no enemy in range. 

\begin{itemize}
    \item If the unit can win the fight and open up a charge in their next
        turn.
    \item If the unit can deal damage while suffering fewer casualties during
        their opponent’s activation.
    \item If it is the only way to prevent enemy unit(s) from moving to
        a better target or capturing an objective in the opponents turn(s).
    \item If the unit can capture an objective.
    \item Units that have charged are always allowed to pile in 3 inch even if
\end{itemize}

Optimizing your Charge move:
\begin{itemize}
    \item When you charge a unit with a single model, you have a clear range
        advantage. While you only need to be in contact with one model, your
        opponent needs to be in range with every model that attacks.
    \item Enemy models may not pile in when they are already in contact with an
        enemy model. Use this to prevent the closest enemy models from making
        room for its buddies by moving around you.
\end{itemize}

\section{Sources}
\href{https://1d4chan.org/wiki/Age_of_Sigmar/Tactics/Order/Daughters_of_Khaine}{Tactics}

\href{https://www.warhammer-community.com/warscroll-builder/}{Battlescroll Builder}

\href{https://whc-cdn.games-workshop.com/wp-content/uploads/2018/06/age_of_sigmar_core_rules_designers_commentary_base_sizes_en.pdf}{FAQ}
